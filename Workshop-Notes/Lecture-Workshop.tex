% Copyright 2007 by Till Tantau
%
% This file may be distributed and/or modified
%
% 1. under the LaTeX Project Public License and/or
% 2. under the GNU Public License.
%
% See the file doc/licenses/LICENSE for more details.



\documentclass{beamer}

% Setup appearance:

\usetheme{}


% Standard packages

\usepackage[english]{babel}
\usepackage{amsmath}
\usepackage{amssymb}
\usepackage{graphicx}
\usepackage{subfigure}
\usepackage{color}
\usepackage{amsbsy}
\usepackage{verbatim}
\setbeamertemplate{footline}[frame number]

% Author, Title, etc.

\title{An introduction to Bayesian modeling \\
using R and JAGS}

\author{
  Instructors \\
  Kent~Holsinger \\
  Xiaojing~Wang % \inst{1}
  \vskip 2mm % \inst{1} \\
  \textit{University of Connecticut}
  \vskip 2mm
  ~~\textit{}
 }

 \date{22 July 2015}




% The main document

\begin{document}

\begin{frame}
  \titlepage
\end{frame}

\begin{frame}{Overview of the Workshop}

\begin{itemize}

\item Focus on using R and JAGS for Bayesian analysis

\item Linear regression including mixed modeling -- Kent Holsinger

\begin{itemize}

\item Simple linear regression

\item Multiple regression (including random effects)

\end{itemize}

\item Multicollinearity -- Xiaojing Wang

\begin{itemize}

\item Hierarchical independent prior distributions

\item Variable selection

\end{itemize}

\end{itemize}

\end{frame}

\begin{frame}{Linear Regression}

One of the most common statistical procedures in ecology and
evolution. For example,

\begin{itemize}
\item Data on LMA from 535 individuals in the genus {\it Protea\/}~(42
  species, 48 sites, 142 unique site/species combinations)
\item Data on mean annual temperature for each of those sites
\end{itemize}

\begin{center}
\includegraphics[height=5.5cm]{lma-vs-mat.pdf}
\end{center}

\end{frame}

\begin{frame}[fragile]{Linear Regression}
In R
{\tiny
\begin{verbatim}
> summary(lm(lma ~ mat, data=tmp))

Call:
lm(formula = lma ~ mat, data = tmp)

Residuals:
      Min        1Q    Median        3Q       Max
-0.025126 -0.005781 -0.000785  0.004647  0.044444

Coefficients:
              Estimate Std. Error t value Pr(>|t|)
(Intercept)  0.0521895  0.0027116  19.246  < 2e-16 ***
mat         -0.0013587  0.0001785  -7.611 1.24e-13 ***
---
Signif. codes:  0 �***� 0.001 �**� 0.01 �*� 0.05 �.� 0.1 � � 1

Residual standard error: 0.009815 on 533 degrees of freedom
Multiple R-squared:  0.09803,	Adjusted R-squared:  0.09634
F-statistic: 57.93 on 1 and 533 DF,  p-value: 1.241e-13
\end{verbatim}
}

\end{frame}

\begin{frame}{Linear Regression}

Remember basic assumptions of simple linear regression

\begin{eqnarray*}
y_i &=& \beta_0 + \beta_1x_i + \epsilon_i \\
\epsilon_i &\sim& \mbox{N}(0, \sigma^2)
\end{eqnarray*}

Here's another way to write that

\begin{eqnarray*}
y_i &\sim& \mbox{N}(\mu_i, \sigma^2) \\
\mu_i &=& \beta_0 + \beta_1x_i
\end{eqnarray*}

The second way of writing the model will be more convenient for us, so
that's the approach we'll use.

\end{frame}

\begin{frame}{Statistical Analysis}
  Statistical inference is the process of learning about the general
  characteristics of a population from a sample.
\begin{itemize}
\item Characteristics often expressed in terms of parameters $\theta$.
\item Measurements on the subset of members given by numerical values $Y$.
\item Before the data are observed, both $Y$ and $\theta$ are unknown.
\item A probability model is assumed for observed data if we knew $\theta$ is the truth.
\item What if we have prior information about $\theta$?
\end{itemize}
\end{frame}

\begin{frame}{Bayesian Inference}
  Bayesian inference allows us to update prior beliefs with the
  observed data to quantify uncertainty about $\theta$.
\begin{itemize}
\item Prior Distribution: $p(\theta)$
\item Sampling Model (likelihood): $p( y \mid \theta)$
\item Posterior Distribution
\[
	p(\theta \mid y)=\frac{p(y \mid \theta) p (\theta)}{p(y)}
\]
\item Calculating $p(y)$ is typically very challenging. Use MCMC
  (implemented in JAGS) to estimate $p(\theta \mid y)$.

\end{itemize}
\end{frame}

\begin{frame}{Metropolis-Hastings Algorithm}
For $\theta_j$
\begin{itemize}
\item propose a new $\theta_j^*\sim q(\theta^{(t)} | \theta_j^t)$
\item Calculate Metropolis-Hastings ratio
\[
\alpha=\frac{p({\bf Y}|\theta^*)p(\theta^*)/q(\theta^*|\theta^{(t)})}{p({\bf Y}|\theta^{(t)})p(\theta^{(t)})/q(\theta^{(t)}|\theta^*)}
\]
\item if $\alpha<1$ set
\[
\theta^{(t+1)}=\left\{
\begin{array}{ll}
\theta^*& \mbox{with probability $\alpha$}\\
\theta^{(t)} & \mbox{with probability $1-\alpha$}
\end{array}
\right.
\]
If $\alpha>1$ set $\theta^{(t+1)}=\theta^{*}$
\end{itemize}

\end{frame}

% \begin{frame}{Metropolis Algorithm}
% Simplest version is called Random-Walk Metropolis
% \begin{itemize}
% \item Take $q(\theta|\theta(t))$ to be a symmetric density
% \item Normal density centered at $\theta^{(t)}$ with variance $c^2$
% \item ratio of proposals cancels!
% \end{itemize}
% \hfill \break
% \texttt{\small{
% thetastar = rnorm(1, theta[t], c)\\
% alpha = min(1, ((logL(thetastar) + logprior(thetastar) -(logL(theta[t]) +logprior(theta[t])))\\
% u = log(runif(1))\\
% theta[t+1] = theta[t]\\
% if (u < alpha) theta[t+1] = thetastar}
% }
% \hfill \break

% Can repeat for each $\theta_j$.
% \end{frame}

\begin{frame}{Linear Regression - as a Bayesian}

We start with the sampling model $p(y \mid \theta)$\footnote{$\theta \in
(\beta_0, \beta_1, \sigma^2)$}

\begin{eqnarray*}
y_i &\sim& \mbox{N}(\mu_i, \sigma^2) \\
\mu_i &=& \beta_0 + \beta_1x_i \quad ,
\end{eqnarray*}

where $x_i$ is the value of the covariate in individual $i$. Then we
add prior distributions $p(\theta)$

\begin{eqnarray*}
\beta_0 &\sim& \mbox{N}(0, \tau) \\
\beta_1 &\sim& \mbox{N}(0, \tau) \\
\sigma^2 &=& \frac{1}{\tau_{resid}} \\
\tau_{resid} &\sim& \mbox{Exponential}(\phi)
\end{eqnarray*}

\end{frame}

\begin{frame}[fragile]{Linear Regression - in R+JAGS}

\begin{itemize}

\item Rescale all variables to mean of 0, standard deviation of 1

{\tiny
\begin{verbatim}
Inference for Bugs model at "simple-linear-regression.jags", fit using jags,
 5 chains, each with 10000 iterations (first 5000 discarded), n.thin = 5
 n.sims = 5000 iterations saved
             mu.vect sd.vect     2.5%      25%      50%      75%    97.5%  Rhat n.eff
beta.0         0.000   0.041   -0.082   -0.028    0.000    0.028    0.080 1.001  5000
beta.mat      -0.313   0.041   -0.393   -0.341   -0.314   -0.285   -0.231 1.002  1900
sigma.resid    0.953   0.029    0.897    0.933    0.952    0.972    1.012 1.001  3700
\end{verbatim}
}
\item Compare with lm() results from R
{\tiny
\begin{verbatim}
              Estimate Std. Error t value Pr(>|t|)
(Intercept)  8.614e-17  4.110e-02   0.000        1
mat         -3.131e-01  4.114e-02  -7.611 1.24e-13 ***

Residual standard error: 0.9506 on 533 degrees of freedom
\end{verbatim}
}
\end{itemize}

\end{frame}

\begin{frame}{Multiple linear regression}

Simple generalization of what we've already seen

\begin{eqnarray*}
y_i &\sim& \mbox{N}(\mu_i, \sigma^2) \\
\mu_i &=& \beta_0 + \sum_{k=1}^K\beta_kx_{ik} \quad ,
\end{eqnarray*}
where $x_{ik}$ is the value of the $k$th covariate in individual
$i$. The priors are
\begin{eqnarray*}
\beta_i &\sim& \mbox{N}(0, \tau), \quad i = 0,\dots,K \\
\sigma^2 &=& \frac{1}{\tau_{resid}} \\
\tau_{resid} &\sim& \mbox{Exponential}(\phi)
\end{eqnarray*}

\end{frame}

\begin{frame}[fragile]{Multiple Linear Regression}

From JAGS
{\tiny
\begin{verbatim}
Inference for Bugs model at "multiple-linear-regression.jags", fit using jags,
 5 chains, each with 10000 iterations (first 5000 discarded), n.thin = 5
 n.sims = 5000 iterations saved
             mu.vect sd.vect     2.5%      25%      50%      75%    97.5%  Rhat n.eff
beta.0         0.000   0.040   -0.078   -0.027    0.000    0.028    0.078 1.001  4800
beta.cdd       0.106   0.073   -0.036    0.058    0.105    0.154    0.251 1.001  5000
beta.elev     -0.319   0.094   -0.500   -0.384   -0.319   -0.256   -0.136 1.001  5000
beta.inso      0.054   0.053   -0.048    0.018    0.054    0.091    0.157 1.001  5000
beta.map      -0.022   0.078   -0.178   -0.074   -0.020    0.031    0.129 1.001  5000
beta.mat      -0.463   0.114   -0.688   -0.541   -0.461   -0.385   -0.243 1.001  5000
beta.ratio    -0.016   0.079   -0.172   -0.069   -0.013    0.039    0.134 1.001  5000
sigma.resid    0.939   0.029    0.884    0.919    0.939    0.958    1.000 1.001  5000
\end{verbatim}
}

Compare to lm() from R
{\tiny
\begin{verbatim}
              Estimate Std. Error t value Pr(>|t|)
(Intercept) -3.520e-17  4.052e-02   0.000 1.000000
cdd          1.051e-01  7.315e-02   1.437 0.151291
elev        -3.184e-01  9.534e-02  -3.339 0.000899 ***
inso         5.288e-02  5.287e-02   1.000 0.317702
map         -2.309e-02  7.749e-02  -0.298 0.765836
mat         -4.629e-01  1.147e-01  -4.037  6.2e-05 ***
ratio       -1.481e-02  7.944e-02  -0.186 0.852218

Residual standard error: 0.9373 on 528 degrees of freedom
\end{verbatim}
}

\end{frame}

\begin{frame}{Multiple Linear Regression with Species Random Effect}

  $\gamma_i^{(s)}$ denotes the mean for species $s$ to which
  inidividual $i$ belongs
\begin{eqnarray*}
y_i &\sim& \mbox{N}(\mu_i, \sigma^2_{resid}) \\
\mu_i &=& \beta_0 + \sum_{k=1}^K\beta_kx_{ik} + \gamma_i^{(s)} \\
\beta_i &\sim& \mbox{N}(0, \tau), \quad i = 0,\dots,K \\
\sigma^2_{resid} &=& \frac{1}{\tau_{resid}} \\
\tau_{resid} &\sim& \mbox{Exponential}(\phi) \\
\gamma_i^{(s)} &\sim& \mbox{N}(0, \sigma^2_{species}) \\
\sigma^2_{species} &=& \frac{1}{\tau_{species}} \\
\tau_{species} &\sim& \mbox{Exponential}(\phi)
\end{eqnarray*}

\end{frame}

\begin{frame}{Multiple Linear Regression with Species Random Effect}

Alternatively
\begin{eqnarray*}
y_i &\sim& \mbox{N}(\mu_i, \sigma^2_{resid}) \\
\mu_i &=& \beta_{0i}^{(s)} + \sum_{k=1}^K\beta_kx_{ik} \\
\sigma^2_{resid} &=& \frac{1}{\tau_{resid}} \\
\tau_{resid} &\sim& \mbox{Exponential}(\phi) \\
\beta_{0i}^{(s)} &\sim& \mbox{N}(\beta_0, \sigma^2_{species}) \\
\sigma^2_{species} &=& \frac{1}{\tau_{species}} \\
\tau_{species} &\sim& \mbox{Exponential}(\phi) \\
\beta_i &\sim& \mbox{N}(0, \tau), \quad i = 0,\dots,K \\
\end{eqnarray*}

\end{frame}

\begin{frame}[fragile]{Multiple Linear Regression with Species Random
    Effect}

From JAGS
{\tiny
\begin{verbatim}
beta.cdd        0.081   0.055  -0.024   0.045   0.082   0.118   0.188 1.001  3100
beta.elev      -0.201   0.108  -0.408  -0.274  -0.201  -0.128   0.010 1.002  2400
beta.inso      -0.073   0.058  -0.187  -0.112  -0.073  -0.034   0.040 1.001  3400
beta.map       -0.418   0.083  -0.581  -0.474  -0.419  -0.362  -0.257 1.001  5000
beta.mat        0.093   0.110  -0.117   0.020   0.092   0.169   0.307 1.001  5000
beta.ratio      0.427   0.078   0.278   0.374   0.427   0.478   0.578 1.001  4800
beta.zero      -0.064   0.154  -0.369  -0.167  -0.062   0.037   0.241 1.001  5000
sigma.resid     0.545   0.023   0.511   0.532   0.544   0.556   0.581 1.001  3100
sigma.species   0.966   0.117   0.767   0.884   0.960   1.036   1.218 1.001  5000
\end{verbatim}
}
Compare to lmer() From R
{\tiny
\begin{verbatim}
Random effects:
 Groups   Name        Variance Std.Dev.
 species  (Intercept) 0.8951   0.9461
 Residual             0.2924   0.5408
Number of obs: 535, groups:  species, 42

Fixed effects:
            Estimate Std. Error t value
(Intercept) -0.06289    0.14898  -0.422
cdd          0.07941    0.05576   1.424
elev        -0.19835    0.10946  -1.812
inso        -0.07259    0.05791  -1.254
map         -0.42009    0.08176  -5.138
mat          0.09811    0.10893   0.901
ratio        0.43022    0.07776   5.533
\end{verbatim}
}

\end{frame}

\begin{frame}[fragile]{Multiple Linear Regression with Species Random
    Effect}

\includegraphics[height=8cm]{species-random-effect.pdf}

\end{frame}


% \begin{frame}{Parametric Sampling Models}
% Assume
% \[
% \left\{
%                 \begin{array}{ll}
%                  Y_1, \cdots, Y_n \stackrel{i.i.d.} \sim p(y \mid \theta)   \\
%                  \theta \sim p(\theta)
%                  \end{array}
%               \right.
% \]
% Applicable if $Y_1, \cdots,  Y_n$ are
% \begin{itemize}
% \item outcomes of a repeatable experiment;
% \item random sample from finite population with
% replacement;
% \item sampled from an infinite population w/out replacement;
% \item sampled from a finite population of size $N \gg n$ w/out
% replacement (approximate).
% \end{itemize}
% Labels carry no information.
% \end{frame}

% \begin{frame}{Normal Model}
% Assume
% \[
% Y_1, \cdots, Y_n \mid \mu, \sigma^2 \stackrel{i.i.d.} \sim \mathcal{N}(\mu, \sigma^2),
% \]
% where $\mu$ and $\sigma^2$ are unknown parameters. From a Bayesian perspective, it is easier to work with the {\it precision}, $\phi$, where $\phi=1/\sigma^2$. Define ${Y}=(Y_1, \cdots, Y_n)$.
% \begin{block}{Likelihood}
% \begin{eqnarray*}
% \mathcal{L} (\mu, \phi \mid { Y}) &\propto& \prod_{i=1}^n \frac{1}{\sqrt{2\pi}}\phi^{1/2} \exp\left\{-\frac{1}{2}\phi(Y_i-\mu)^2\right\}\\
% &\propto& \phi^{n/2} \exp\left\{-\frac{1}{2} \phi \sum_{i=1}^n (Y_i-\mu)^2\right\}.
% \end{eqnarray*}
% \end{block}
% \end{frame}

% \begin{frame}{Likelihood Factorization}
% \vskip -0.25in
% \begin{eqnarray*}
% \mathcal{L} (\mu, \phi \mid {Y})
% &\propto& \phi^{n/2} \exp\left\{-\frac{1}{2} \phi \sum_{i=1}^n (Y_i-\mu)^2\right\}\\
% &\propto& \phi^{n/2} \exp\left\{-\frac{1}{2} \phi\sum_{i=1}^n \left[(Y_i-\overline{Y})-(\mu-\overline{Y})\right]^2\right\}\\
%  &\propto& \phi^{n/2} \exp\left\{-\frac{1}{2} \phi \left[\sum_{i=1}^n(Y_i-\overline{Y})^2+n(\mu-\overline{Y})^2\right]^2\right\}\\
%   &\propto& \phi^{n/2} \exp\left\{-\frac{1}{2} \phi s^2 (n-1)\right\} \exp\left\{\frac{1}{2}\phi n(\mu-\overline{Y})^2\right\}\\
%    &\propto& \phi^{n/2} \exp\left\{-\frac{1}{2} \phi SS\right\} \exp\left\{\frac{1}{2}\phi n(\mu-\overline{Y})^2\right\},
% \end{eqnarray*}
% where $\overline{Y}=\sum_{i=1}^n Y_i/n$ is the sample mean, $s^2=\sum_{i=1}^n (Y_i -\overline{Y})^2/(n-1)$ is the sample variance and $SS=\sum_{i=1}^n (Y_i-\overline{Y})^2$ is the sample sum of squares.
% \end{frame}

% \begin{frame}
% \begin{block}{Conjugate Priors}
% Consider a class of prior distributions, $p(\theta) \in \mathcal{P}$.  We say that the class is conjugate for a sampling model $p(Y \mid \theta)$, if $p(\theta) \in \mathcal{P}$ implies that $p(\theta \mid y) \in  \mathcal{P}$ for all $p(\theta) \in \mathcal{P}$ and data $y$.
% \end{block}
% \begin{block}{Conjugate Normal-Gamma Prior for Normal Data}
% The conjugate prior distribution for $(\mu, \phi)$ is Normal-Gamma, i.e.,
% \begin{eqnarray*}
% \mu \mid \phi & \sim & \mathcal{N} (m_0, 1/(p_0\phi)), \\
% \phi & \sim & \mathcal{G}a (\nu_0/2, SS_0/2),
% \end{eqnarray*}
% which is written as
% \[
% 	p(\mu, \phi)\propto \phi^{\nu_0/2-1}\exp\left\{-\phi\frac{SS_0}{2}\right\} \exp\left\{-\phi\frac{p_0}{2}(\mu-m_0)^2\right\}.
% \]
% and further we can denote it is drawn from a Normal-Gamma family
% \[
% 	\mu, \phi \sim \mathcal{NG}(m_0, p_0, \nu_0/2, SS_0).
% \]
% \end{block}
% \end{frame}

% \begin{frame}{Updating the Posterior Parameters}
% Under the Normal-Gamma prior distribution, we can derive the posterior distribution
% \begin{eqnarray*}
% \mu \mid \phi, Y &\sim& \mathcal{N}\left(m_n, \frac{1}{p_n\phi}\right), \\
% \phi \mid Y &\sim & \mathcal{G}a\left(\frac{\nu_n}{2}, \frac{SS_n}{2}\right), \\
% \mbox{where} \  p_n&=&p_0+n, \\
% m_n&=&\frac{n\overline{Y}+p_0m_0}{p_n}, \\
% \nu_n &=& \nu_0+n,  \\
% SS_n &=& SS_0+SS+\frac{np_0}{p_n}(\overline{Y}-m_0)^2.
% \end{eqnarray*}
% The posterior distribution of $(\mu, \phi)$ can be rewritten as a Normal-Gamma family
% \[
%    \mu, \phi \mid Y \sim \mathcal{NG}(m_n, p_n, \nu_n/2, SS_n/2).
% \]
% \end{frame}

% \begin{frame}{Interpretation}
% \begin{itemize}
% \item $p_n$ indicates the precision for estimating $\mu$ after getting $n$ observations.
% \item $m_n$ is the expected value for $\mu$ after obtaining $n$ observations,  which can be viewed as a weighted average of sample mean and prior mean, i.e.,
% \[
% 	m_n=\frac{n}{p_n}\overline{Y}+\frac{p_0}{p_n} m_0.
% \]
% \item $\nu_n$ is called the degrees of freedom, by noticing that
% \[
% 	\phi \sim \mathcal{G}a(a/2, b/2)\Leftrightarrow \phi b \sim \chi_a^2\ \mbox{with degrees of freedom} \ a.
% \]
% \item Denote $SS_n=SS_0+SS+\frac{np_0}{p_n}(\overline{Y}-m_0)^2$ as the posterior variation, where the three terms can be explained as prior variation, observed variation (sum of squares) and variation between prior mean and sample mean, respectively.
% \end{itemize}
% \end{frame}

% \begin{frame}{Marginal Distribution for $\mu \mid Y$}
% \begin{eqnarray*}
% p(\mu |Y) &\propto& \int{p(\mu,\phi | Y)d\phi}\\
% &=&\int{\phi^{\frac{\nu_n+1}{2}-1}}\exp\left[-\phi\left\{\frac{SS_n+p_n(\mu-m_n)^2}{2}\right\}\right]d\phi
% \end{eqnarray*}

% This has the form of a Gamma integral with $a = (\nu+ 1)/2$ and $b$ equal to the mess multiplying $\phi$ in the exponential term, so that the result is $\propto b^{-a}$ (at least that is all that matters)
% \begin{eqnarray*}
% p(\mu \mid Y) &\propto& \left\{  SS_n+p_n(\mu-m_n)^2  \right\}^{\frac{-(\nu_n+1)}{2}}\\
% &\propto& \left( \nu_n+\frac{(\mu-m_n)^2}{\frac{1}{p_n}\frac{SS_n}{\nu_n}} \right)^{-(\nu_n+1)/2},
% \end{eqnarray*}
% which is a Student $t_{\nu_n}(m_n,s_n^2)$ with location $m_n,\text{d}f=\nu_n$, scale $s_n^2=\frac{1}{p_n}\frac{SS_n}{\nu_n}$ and degrees of freedom being $\nu_n$.
% \end{frame}


% \begin{frame}{Standard Student $t$}
% Standardize $X\sim t_{\delta}(l,S)$ by subtracting location and dividing by square root of the scale:
% \[
% \frac{X-l}{\sqrt{S}}\sim t_{\delta}(0,1)
% \]
% (new location 0 and scale 1)
% \[
% \Rightarrow \frac{\mu-m_n}{s_n}\sim t_{\nu_n}(0,1)
% \]
% \[
% \mu \overset{D}{=}m_n+t_{\nu_n}s_n,
% \]
% where $t_{\nu_n}$ can be easily evaluated using $rt$, $qt$, $pt$, and $dt$ in R.
% \end{frame}

% \begin{frame}{Problem}
% \begin{itemize}
% \item Data: Observe pairs $(Y_i, X_i)$, $i=1, \cdots, n$;
% \item Response or dependent variable $Y$;
% \item Predictor or independent variable $X$.
% \end{itemize}
% \begin{block}{Goals}
% \begin{itemize}
% \item Exploring $p(y \mid x)$ as a function of $x$;
% \item Understanding the mean (variability) in $Y$ as a function of $X$;
% \item Special case: Linear regression (normal $Y$).
% \end{itemize}
% \end{block}
% \end{frame}

% \begin{frame}{Simple Linear Regression Models}
% \[
% Y_i=\alpha+\beta X_i+\epsilon_i,  \epsilon_i \stackrel{i.i.d.}{\sim} N(0,\sigma^2)
% \]
% \begin{itemize}
% \item Estimate parameters ($\alpha$,$\beta$,$\sigma^2$);
% \item interpretation of parameters: $\beta$, $\alpha$
% \item assess model fit -- adequate? good? if inadequate, how?
% \item predict new (``future'') response at new $x_{n+1},\cdots$
% \item how much variability does $x$ explain?
% \end{itemize}
% \end{frame}

% \begin{frame}{Conjugate Priors for Simple Linear Regression}
% \begin{block}{Normal-Gamma Prior}
% The Normal-Gamma distribution is conjugate for $\alpha$, $\beta$ and precision $\phi\equiv{1}/{\sigma^2}$.
% \begin{itemize}
% \item $(\alpha,\beta)|\phi \sim N((\alpha_0,\beta_0), \phi^{-1}\Sigma)$,  where  $\Sigma$ is a $2\times 2$ matrix of variances and covariance;
% \item $\phi \sim \text{Gamma}(\nu_0/2, \nu_0\sigma^2_0/2)$.
% \end{itemize}
% \end{block}
% \begin{block}{A Reference Prior for Regression}
% Limiting case of conjugate prior as the prior variances goes to infinity (information goes to zero)
% \[
% p(\alpha,\beta,\phi) \propto 1/\phi.
% \]
% \end{block}

% \end{frame}

% \begin{frame}{Theory for Inference}
% \[
% \frac{\beta-\hat{\beta}}{\sqrt{s^2_{Y|X}\frac{1}{S_{xx}}}}\sim t_{n-2}(0,1)
% \]

% \begin{itemize}
% \item $\hat{\beta}$ is OLS(MLE) estimate of $\beta$, $s^2_{Y|X}=\hat{\sigma}^2$ is the MSE.
% \item (marginal) posterior for $\beta$ is a Student $t$ distribution with $n-2$ df.
% \item Sampling distribution of $\hat{\beta}$ given $\beta$ is Student $t$ distribution with $n-2$ df.
% \end{itemize}

% Used for classical and Bayesian (Reference) analysis

% \end{frame}

% \begin{frame}{Distributions Continued}
% \begin{itemize}
% \item (marginal) posterior for $\alpha$ is a Student $t$ distribution with $n-2$ df.
% \item Sampling distribution of $\hat{\alpha}$ is Student $t$ distrbution with $n-2$ df.
% \end{itemize}

% \[
% \frac{\alpha-\hat{\alpha}}{\sqrt{s^2_{Y|X}\left(\frac{1}{n}+\frac{\bar{x}^2}{S_{xx}}\right)}}\sim t_{n-2}(0,1)
% \]

% \end{frame}

% \begin{frame}{Significance of Regression}
% Measuring the ''explanatory power'' of predictor $X$
% \begin{itemize}
% \item Credible Intervals (HPD or equal-tailed)
% \[
% \hat{\beta}\pm t_{\alpha/2} S_{\beta}
% \]
% where $t_{\alpha/2}$ is 100 $\alpha/2\%$ quantile of a standard $t_{n-2}$ and $s_{\beta}=\sqrt{s^2_{Y|X}/S_{xx}}$
% \item Confidence Intervals:
% \[
% \hat{\beta}\pm t_{\alpha/2} SE_{\hat{\beta}}
% \]
% and $SE(\hat{\beta})=\sqrt{s^2_{Y|X}1/S_{xx}}$
% \end{itemize}
% \end{frame}

% \begin{frame}{Predictions}
% The (posterior) predictive distribution for a new case, $y_{n+1}=\alpha+\beta x_{n+1}+\epsilon_{n+1}$ is also a Student $t$ distribution with $n-2$ df.
% \begin{eqnarray*}
% y_{n+1}|y_1,\cdots,y_n &\sim & t_{n-2}(\hat{y},s^2_{y_{n+1}})\\
% \hat{y}&=& \hat{\alpha}+\hat{\beta}x_{n+1}\\
% s^2_{y_{n+1}}&=&s^2_{Y|X}\left(  1+\frac{1}{n}+\frac{(x_{n+1}-\bar{x})^2}{S_{xx}}   \right)
% \end{eqnarray*}

% \begin{itemize}
% \item posterior uncertainty about $\alpha+\beta x_{n+1}$
% \item depends on $x_{n+1}$ spread is higher for $x_{n+1}$ far from $\bar{x}$
% \item additional variability $+s^2_{Y|X}$ due to $\epsilon_{n+1}$
% \end{itemize}

% \end{frame}

% \begin{frame}{Multiple Linear Regression Models}
% Let us assume the general linear model is
% \[
% 	y_i={\bf x}_i'\boldsymbol\beta+\epsilon_i, \ \epsilon_i \sim \mathcal{N}(0, \tau^{-1}),
% \]
% where $\tau$ is called the error precision.
% \begin{itemize}
% \item Likelihood:
% \[
% 	\mathcal{L}({\bf y, X}; \boldsymbol\beta, \tau)=\prod_{i=1}^n \sqrt{\frac{\tau}{2\pi}}\exp\left\{-\frac{\tau}{2}(y_i-{\bf x}_i'\boldsymbol\beta)^2\right\}
% \]
% \item Normal-Gamma Prior:
% \begin{eqnarray*}
% 	\pi(\boldsymbol\beta, \tau)&=&\pi(\boldsymbol\beta \mid \tau)\pi(\tau)=\mathcal{N}_p(\boldsymbol\beta; \boldsymbol\beta_0, \tau^{-1}\Sigma_0)\mathcal{G} (\tau; a, b). \\
% 	\pi (\boldsymbol\beta \mid \tau )&=&| 2\pi \tau^{-1}\Sigma_0 |^{-p/2} \exp\left\{-\frac{\tau}{2}(\beta-\beta_0)'\Sigma_0^{-1}(\boldsymbol\beta-\boldsymbol\beta_0)\right\},
% \end{eqnarray*}
% where $\boldsymbol\beta_0$ is the prior mean and $\Sigma_0$ is the prior covariance and
% \[
% 	\pi(\tau)=\frac{b^a}{\Gamma(a)} \tau^{a-1}\exp\left\{-b\tau\right\}.
% \]
% \item A non-informative prior: $\pi(\boldsymbol\beta, \tau)\propto 1/\tau$.
% \end{itemize}

% \end{frame}

% \begin{frame}{Bayesian Regression with Non-informative Priors}
% \begin{itemize}
% \item The non-informative prior are invalid probabilities
% (it does not integrate to any finite number). So why is it
% that we are even discussing them?
% \item It turns out that even if the priors are improper (that's what
% we call them), as long as the resulting posterior
% distributions are valid we can still conduct legitimate
% statistical inference on them.
% \item When the non-informative prior is used, after some algebra, the joint posterior distribution of $(\boldsymbol\beta, \tau)$ is
% \begin{eqnarray*}
% 	&&\pi(\boldsymbol\beta, \tau \mid {\bf X, y})\\
% 	&\propto& \mathcal{L}({\bf y, X}; \boldsymbol\beta, \tau) \pi(\boldsymbol\beta, \tau) \\
% 	&\propto& (\tau)^{n/2-1} \exp\left(-\frac{\tau}{2}(\hat{\sigma}^2 (n-p)+(\boldsymbol\beta-\hat{\boldsymbol\beta})'{\bf X'X}(\boldsymbol\beta-\hat{\boldsymbol\beta})\right),
% \end{eqnarray*}
% where $
%     \hat{\boldsymbol\beta}=({\bf X'X})^{-1}{\bf X'y} \ \mbox {and} \ \hat{\sigma}^2=\frac{({\bf y-X\hat{\boldsymbol\beta}})'({\bf y-X\hat{\boldsymbol\beta}})}{n-p}$ ,
% which are the classical unbiased estimates of the regression
% parameter $\boldsymbol\beta$ and $\sigma^2=\tau^{-1}$.
% \end{itemize}
% \end{frame}

% \begin{frame}{Bayesian Regression with Non-informative Priors (Cont.)}
% \begin{itemize}
% \item The marginal posterior distribution of $\tau$ is actually a Gamma distribution, i.e., $\tau \mid {\bf X,y}\sim \mathcal{G}((n-p)/2, (n-p)\hat{\sigma}^2/2)$. In other words, $\sigma^2$ follow a inverse Gamma distribution, i.e., $\sigma^2 \mid {\bf X,y} \sim \mathcal{IG}((n-p)/2, (n-p)\hat{\sigma}^2/2)$.  That implies $(n-p)\hat{\sigma}^2/\sigma^2 \mid {\bf X, y} \sim \chi_{n-p}^2$.
% \item A striking similarity with the classic result: The distribution of $\hat{\sigma}^2$  is also characterized as $(n-p)\hat{\sigma}^2/\sigma^2$ following a chi-square distribution with $n-p$ degrees of freedom.
% \item The marginal posterior distribution of $\boldsymbol\beta$ follows a multivariate $t$ distribution, i.e.,
% \begin{eqnarray*}
% 	\pi(\boldsymbol\beta \mid {\bf X, y})&=&\frac{\Gamma (n/2)}{\pi^{p/2} (n-p)^{p/2}\Gamma((n-p)/2) s^p|({\bf X'X})|^{-1/2}}\\
% 	&\times&\left[1+\frac{(\boldsymbol\beta-\hat{\boldsymbol\beta})'{\bf X'X}(\boldsymbol\beta-\hat{\boldsymbol\beta})}{(n-p)\hat{\sigma}^2}\right]^{-n/2},
% \end{eqnarray*}
% which we denote as $\boldsymbol\beta \mid {\bf X, y}\sim \mathcal{{\bf t}}_{n-p} (\hat{\boldsymbol\beta}, \hat{\sigma}^2 ({\bf X'X})^{-1})$.
% \end{itemize}
% \end{frame}

% \begin{frame}{Interpretation}
% Under the non-informative prior $\pi(\boldsymbol\beta, \tau)\propto 1/\tau$,
% \begin{itemize}
% \item Marginally $\beta_j \mid {\bf X,y} \sim t_{n-p}(\hat{\beta}_j, s_{\beta_j}^2)$, where $s_{\beta_j}^2$ is the $j$th element on the diagonal of $\hat{\sigma}^2 ({\bf X'X})^{-1}$.
% \item  HPD interval $\hat{\beta}_j \pm t_{n-p, \alpha/2} s_{\beta_j}$.
% \end{itemize}
% Question: How probable is $\beta_j = 0$ under the posterior?
% \begin{itemize}
% \item Informal ``test'': Probability in tails = signifcance level =
% (Bayesian) p-value
% \[
%        \mbox{$p$-value}={\rm P}(|t| > |\hat{\beta}_j/s_{{\beta}_j}|).
% \]
% \item Lindley's Method: Lindley suggested rejecting the hypothesis that $\beta_j=0$ at the $\alpha$ level of significance if the $100(1-\alpha)\%$ HPD region does not include $0$, i.e.,
% \[
% 	0\notin (\hat{\beta}_j - t_{n-p, \alpha/2} s_{\beta_j}, \hat{\beta}_j +t_{n-p, \alpha/2} s_{\beta_j}).
% \]
% which is equivalent to comparing the p-value to $\alpha$ and concluding that the
% regression is significant if the p-value is less than $\alpha$.
% \end{itemize}
% \end{frame}



% \begin{frame}{Bayes Factor}
% Testing $H_0: \beta_j=0$ versus $H_a:\beta_j\neq 0$.
% \begin{itemize}
% \item Assign prior probabilities to $H_0$ and $H_a$.
% \item Find ${\rm P}(H_i \mid {\bf y} )$ via Bayes Theorem.
% \end{itemize}
% Bayes Factor for comparing evidence in favor of $H_0$
% \[
% 	BF[H_0: H_a]=\frac{P(H_0 \mid {\bf y})/p(H_0)}{P(H_a \mid {\bf y})/p(H_a)}.
% \]
% Often difficult to calculate, instead use lower bound based on p-values (Berger, Sellke and Bayarri, 2001)
% \[
% 	BF[H_0: H_a]=-ep\log(p).
% \]
% \end{frame}

% \begin{frame}{Jeffreys Scale of Evidence}
% \begin{center}
% \includegraphics[scale=0.5]{BFTABLE}
% \end{center}
% Here $B=BF[H_0: H_a]$.
% \end{frame}


\begin{frame}{Problems with Multicollinearity}
\begin{itemize}
\item Variables may appear to be unimportant (when they are).
\item  Coefficient estimates are unstable and hard to interpret (can estimate combinations of coefficients but not individual coefficients).
\end{itemize}

Alternative Bayesian solutions:
\begin{itemize}
\item Independent Prior Distributions
\item Variable Selection
\end{itemize}
\end{frame}

\begin{frame}{Hierarichal Model with Independent Priors}
Hierarchical Model:
\begin{equation*}
\begin{aligned}
\beta_j|\lambda_j,\sigma^2 &\sim \mbox{N}(0,\sigma^2/\lambda_j)\\
 \lambda_j|\sigma^2 &\sim \mbox{Gamma}(1/2,1/2)\\
1/\sigma^2 &\sim \mbox{Gamma}(\nu_0/2,\nu_0\sigma_0^2/2)
\end{aligned}
\end{equation*}

\begin{itemize}
% \item leads to nice conjugate updates for all full conditionals.
\item {\color{red}{Xiaojing: Add brief statement of why this might reduce
      problem of collinearity relative to independent Normal priors}}
\item Easy to code in JAGS
\item Allows each parameter to have own precision with mean 1
\end{itemize}

\end{frame}

\begin{frame}{Cauchy Prior}
First two equations imply that $\beta_j |\sigma^2 \sim \mbox{Cauchy}(0,\sigma^2)$
\[
p(\beta)=\frac{1}{\pi\sigma}\left(1+\frac{\beta^2}{\sigma^2} \right)^{-1}
\]
leading to a collapsed model
\begin{equation*}
\begin{aligned}
{\bf Y}|\beta,\sigma^2 &\sim \mbox{N}({\bf X\beta},\sigma^2{\bf I_n})\\
\beta_j|\sigma^2 &\sim \mbox{Cauchy}(0,\sigma^2)\\
1/\sigma^2 &\sim \mbox{Gamma}(\nu_0/2,\nu_0\sigma_0^2/2)
\end{aligned}
\end{equation*}
No nice full conditional for $\beta_j$.
\end{frame}

\begin{frame}[fragile]{Cauchy Prior}

Independent $\mbox{N}(0, 1)$
{\tiny
\begin{verbatim}
beta.cdd        0.081   0.055  -0.024   0.045   0.082   0.118   0.188 1.001  3100
beta.elev      -0.201   0.108  -0.408  -0.274  -0.201  -0.128   0.010 1.002  2400
beta.inso      -0.073   0.058  -0.187  -0.112  -0.073  -0.034   0.040 1.001  3400
beta.map       -0.418   0.083  -0.581  -0.474  -0.419  -0.362  -0.257 1.001  5000
beta.mat        0.093   0.110  -0.117   0.020   0.092   0.169   0.307 1.001  5000
beta.ratio      0.427   0.078   0.278   0.374   0.427   0.478   0.578 1.001  4800
beta.zero      -0.064   0.154  -0.369  -0.167  -0.062   0.037   0.241 1.001  5000
sigma.resid     0.545   0.023   0.511   0.532   0.544   0.556   0.581 1.001  3100
sigma.species   0.966   0.117   0.767   0.884   0.960   1.036   1.218 1.001  5000
\end{verbatim}
}
Independent hierarchical
{\tiny
\begin{verbatim}
beta.cdd        0.084   0.055  -0.026   0.048   0.085   0.122   0.188 1.003  1300
beta.elev      -0.191   0.104  -0.394  -0.262  -0.190  -0.120   0.009 1.004   800
beta.inso      -0.077   0.059  -0.192  -0.117  -0.077  -0.038   0.039 1.001  5000
beta.map       -0.400   0.082  -0.562  -0.455  -0.401  -0.344  -0.240 1.004   870
beta.mat        0.084   0.107  -0.129   0.013   0.084   0.157   0.288 1.009   370
beta.ratio      0.416   0.079   0.262   0.363   0.417   0.469   0.570 1.007   510
beta.zero      -0.060   0.142  -0.338  -0.154  -0.061   0.035   0.224 1.002  2700
sigma.resid     0.545   0.018   0.512   0.533   0.544   0.556   0.580 1.001  5000
sigma.species   0.958   0.114   0.767   0.876   0.947   1.028   1.211 1.001  5000
\end{verbatim}
}

\end{frame}

\begin{frame}{Stochastic Search Variable Selection}
The Spike-and-Slab prior:
\begin{eqnarray*}
	\beta_j \mid \gamma_j, c, \tau_j^2 &\sim& (1-\gamma_j)\mathcal{N}(0, \tau_j^2)+\gamma_j \mathcal{N}(0, \tau_j^2 c^2) \\
	\gamma_j \mid \pi_j &\sim& Bernoulli(\pi_j)
\end{eqnarray*}
\begin{columns}
\begin{column}{0.5\textwidth}
        \includegraphics[scale=0.75]{SAS}
    \end{column}
    \begin{column}{0.47\textwidth}
        \begin{itemize}
            \item $\gamma_j=0$: Variable not in the model;
            \item $\gamma_j=1$:  Variable in the model;
            \item  Calibration of hyper-parameters $c$, $\tau_j^2$ needed.
        \end{itemize}
    \end{column}
\end{columns}
\end{frame}

\begin{frame}
\begin{block}{Inference for Variable Selection}
\begin{itemize}
\item Highest posterior model (HPM): Select a model that has been visited most often.
\item Median probability model (MPM): Select variables that appear at least in 50\% of visited models.
\end{itemize}
\end{block}
\begin{block}{Alternative spike and slab models}
\begin{itemize}
\item Popular approach in genomic research;
\item Variants:
\begin{itemize}
\item Conjugate version:
\[
	\beta_j \mid \gamma_j, c, \tau_j^2 \sim (1-\gamma_j)\mathcal{N}(0, \sigma^2\tau_j^2)+\gamma_j \mathcal{N}(0, \sigma^2 \tau_j^2 c^2).
\]
\item Replace the spike normal in Spike-and-Slab prior by Dirac, i.e.,
\[
      \beta_j \mid \gamma_j, \tau_j^2 \sim (1-\gamma_j)\delta_0+\gamma_j \mathcal{N}(0, \tau_j^2).
\]
\end{itemize}
\end{itemize}
\end{block}
\end{frame}

\begin{frame}[fragile]{Variable selection -- Dirac + Normal}

Independent $\mbox{N}(0, 1)$
{\tiny
\begin{verbatim}
beta.cdd        0.081   0.055  -0.024   0.045   0.082   0.118   0.188 1.001  3100
beta.elev      -0.201   0.108  -0.408  -0.274  -0.201  -0.128   0.010 1.002  2400
beta.inso      -0.073   0.058  -0.187  -0.112  -0.073  -0.034   0.040 1.001  3400
beta.map       -0.418   0.083  -0.581  -0.474  -0.419  -0.362  -0.257 1.001  5000
beta.mat        0.093   0.110  -0.117   0.020   0.092   0.169   0.307 1.001  5000
beta.ratio      0.427   0.078   0.278   0.374   0.427   0.478   0.578 1.001  4800
beta.zero      -0.064   0.154  -0.369  -0.167  -0.062   0.037   0.241 1.001  5000
sigma.resid     0.545   0.023   0.511   0.532   0.544   0.556   0.581 1.001  3100
sigma.species   0.966   0.117   0.767   0.884   0.960   1.036   1.218 1.001  5000
\end{verbatim}
}
Dirac + $\mbox{N}(0,1)$
{\tiny
\begin{verbatim}
beta.cdd        0.008   0.037   0.000   0.000   0.000   0.000   0.161 1.128   120
beta.elev      -0.240   0.138  -0.450  -0.339  -0.274  -0.178   0.000 1.072    56
beta.inso      -0.005   0.023  -0.087   0.000   0.000   0.000   0.000 1.036   490
beta.map       -0.443   0.120  -0.603  -0.515  -0.466  -0.406   0.000 1.074    86
beta.mat        0.018   0.085   0.000   0.000   0.000   0.000   0.290 1.053    93
beta.ratio      0.373   0.079   0.238   0.338   0.373   0.409   0.539 1.026   290
beta.zero      -0.058   0.149  -0.346  -0.159  -0.057   0.037   0.234 1.001  5000
sigma.resid     0.548   0.024   0.514   0.535   0.547   0.559   0.586 1.004   940
sigma.species   0.941   0.117   0.739   0.861   0.935   1.012   1.188 1.005   700
\end{verbatim}
}

\end{frame}

\begin{frame}[fragile]{Variable selection -- Dirac + Normal}

Dirac + $\mbox{N}(0,1)$
{\tiny
\begin{verbatim}
beta.cdd        0.008   0.037   0.000   0.000   0.000   0.000   0.161 1.128   120
beta.elev      -0.240   0.138  -0.450  -0.339  -0.274  -0.178   0.000 1.072    56
beta.inso      -0.005   0.023  -0.087   0.000   0.000   0.000   0.000 1.036   490
beta.map       -0.443   0.120  -0.603  -0.515  -0.466  -0.406   0.000 1.074    86
beta.mat        0.018   0.085   0.000   0.000   0.000   0.000   0.290 1.053    93
beta.ratio      0.373   0.079   0.238   0.338   0.373   0.409   0.539 1.026   290
beta.zero      -0.058   0.149  -0.346  -0.159  -0.057   0.037   0.234 1.001  5000
sigma.resid     0.548   0.024   0.514   0.535   0.547   0.559   0.586 1.004   940
sigma.species   0.941   0.117   0.739   0.861   0.935   1.012   1.188 1.005   700
\end{verbatim}
}
Posterior conditioned on $\gamma_i > 0$
{\tiny
\begin{verbatim}
  beta.cdd:  0.06      0.121 (-0.028,  0.279)
 beta.elev:  0.81     -0.296 (-0.457, -0.126)*
 beta.inso:  0.06     -0.077 (-0.183,  0.031)
  beta.map:  0.96     -0.460 (-0.604, -0.259)*
  beta.mat:  0.14      0.135 (-0.351,  0.407)
beta.ratio:  0.99      0.379 ( 0.264,  0.540)*
\end{verbatim}
}
\end{frame}


\begin{frame}{Model Selection}
Selection of a single model has the following problems
\begin{itemize}
\item When the criteria suggest that several models are equally
good, what should we report? Still pick only one model?
\item What do we report for our uncertainty after selecting a model?
\end{itemize}
Typical analysis ignores model uncertainty!
\end{frame}

\begin{frame}{Bayesian Model Choice}
\begin{itemize}
\item Models for the variable selection problem are based on a
subset of the ${\bf x}_1, \cdots, {\bf x}_p$ variables.
\item Encode models with a vector $\boldsymbol\gamma=(\gamma_1, \cdots, \gamma_p)'$ where $\gamma_j \in \{0,1\}$ is is an indicator for whether variable ${\bf x}_j$ should be
included in the model $M_{\boldsymbol\gamma}$.  Notice $\gamma_j=0\Leftrightarrow \beta_j=0$.
\item Each value of $\boldsymbol\gamma$ represents one of the $2^p$ models.
\item Under model $M_{\boldsymbol\gamma}$:
\[
	{\bf y} \mid \boldsymbol\beta, \boldsymbol\gamma, \tau \sim \mathcal{N}({\bf X}_{\boldsymbol\gamma}\boldsymbol\beta_{\boldsymbol\gamma}, \tau^{-1}{\bf I})
\]
where ${\bf X}_{\boldsymbol\gamma}$ is design matrix using the columns in $\bf X$ where $\gamma_j=1$ and $\boldsymbol\beta_{\boldsymbol\gamma}$ is the subset of $\boldsymbol\beta$ that are non-zero.
\end{itemize}
\end{frame}

\begin{frame}{Bayesian Model Averaging}
Rather than use a single model, BMA uses all (or potentially a lot)
models, but weights model predictions by their posterior
probabilities (measure of how much each model is supported by
the data).
\begin{itemize}
\item Posterior model probabilities
\[
	{\rm P}(M_j \mid {\bf y})=\frac{{\rm P}({\bf y}\mid M_j){\rm P}(M_j)}{\sum_{j}{\rm P}({\bf y}\mid M_j){\rm P}(M_j)},
\]
Marginal likelihod of a model is
\[
	{\rm P}({\bf y}\mid M_{\boldsymbol\gamma})=\int \int {\rm P}({\bf y} \mid \boldsymbol\beta_{\boldsymbol\gamma}, \tau) {\rm P}(\boldsymbol\beta_{\boldsymbol\gamma} \mid \boldsymbol\gamma, \tau) {\rm P}(\tau \mid \boldsymbol\gamma) d\boldsymbol\beta_{\boldsymbol\gamma} d\tau.
\]
\item Probability $\beta_j\neq 0$: $\sum_{M_j: \beta_j\neq 0} {\rm P}(M_j \mid {\bf y})$.
\end{itemize}
\end{frame}

\begin{frame}{Bayesian Model Averaging (Continued)}
\begin{itemize}
\item Predictions
\[
	{\rm P}({\bf y}^{new} \mid {\bf y})=\sum_{j} {\rm P}({\bf y}^{new} \mid {\bf y}, M_j) {\rm P}(M_j \mid {\bf y}),
\]
where
\[
	{\rm P}({\bf y}^{new} \mid {\bf y}, M_{\boldsymbol\gamma})=\int {\rm P}({\bf y}^{new} \mid {\bf y}, \boldsymbol\beta_{\boldsymbol\gamma}, \tau) {\rm P}(\boldsymbol\beta_{\boldsymbol\gamma}, \tau \mid {\bf y}) d \boldsymbol\beta_{\boldsymbol\gamma} d\tau.
\]

\end{itemize}
\end{frame}

% \begin{frame}{Prior Distributions}
% \begin{itemize}
% \item Bayesian Model choice requires proper prior distributions on
% regression coefficients.
% \item Vague but proper priors may lead to paradoxes!
% % \item Conjugate Normal-Gammas lead to closed form expressions
% % for marginal likelihoods, Zellner's g-prior is the most popular.
% \end{itemize}
% \end{frame}

% \begin{frame}{Zellner's g-prior}
% Centered model:
% \[
% 		{\bf y}={\bf 1}_n \alpha+ \tilde{\bf X}_c\boldsymbol\beta+\boldsymbol\epsilon,
% \]
% where $\tilde{\bf X}_c=({\bf I}_n-1/n {\bf J}_n){\bf X}$ is the centered design matrix where all variables have had their mean subtracted.
% \begin{itemize}
% \item $\pi(\alpha)\propto 1$;
% \item $\pi(\tau)\propto 1/\tau$;
% \item $\boldsymbol\beta \mid \tau, \boldsymbol\gamma \sim \mathcal{N}({\bf 0}, g\tau^{-1} (\tilde{\bf X}_c'\tilde{\bf X}_c)^{-1})$;
% \item take $g=n$.
% \end{itemize}
% which leads to marginal likelihood of $M_{\boldsymbol\gamma}$ that is proportional to
% \[
% 	{\rm P}({\bf y} \mid M_{\boldsymbol\gamma}, g)\propto \frac{(1+g)^{(n-p_{\boldsymbol\gamma}-1)/2}}{(1+g(1-R_{\boldsymbol\gamma}^2))^{(n-1)/2}},
% \]
% where $R_{\boldsymbol\gamma}^2$ is the ordinary coefficient of determination of regression model $M_{\boldsymbol\gamma}$.

% Lastly, assign uniform distribution to space of models.
% \end{frame}



% \begin{frame}{Air Pollution Data}
% \begin{itemize}
% \item Response $SO_2$ measurements in $41$ metropolitan areas.
% \item Predictors:
% \begin{itemize}
% \item temp
% \item mfgfirms
% \item popn
% \item wind
% \item precip
% \item raindays
% \end{itemize}
% Model for $SO_2$ as a function of the other variables?
% \end{itemize}
% \end{frame}

% \begin{frame}{Scatterplot Matrix}
% Original Variables\\
% \begin{center}
% \includegraphics[scale=0.5]{SMP}
% \end{center}
% \end{frame}

% \begin{frame}{Scatterplot Matrix}
% Transformed Predictor Variables\\
% \begin{center}
% \includegraphics[scale=0.275]{p_22}
% \end{center}
% \end{frame}

% \begin{frame}{Residual Plots}
% \begin{center}
% \includegraphics[scale=0.275]{p_21}
% \end{center}
% \end{frame}

% \begin{frame}{BoxCox Profile Likelihood}
% \begin{center}
% \includegraphics[scale=0.275]{p_bc}
% \end{center}
% $\lambda\approx-0.5$.
% \end{frame}

% \begin{frame}{Residual Plots}
% \begin{center}
% \includegraphics[scale=0.275]{p_23}
% \end{center}
% \end{frame}

% \begin{frame}
% \begin{table}[ht]
% \centering
% \caption{Least Squares Estimates of the Coefficients}
% \begin{tabular}{rrrrr}
%   \hline
%  & Estimate & Std. Error & t value & Pr($>$$|$t$|$) \\
%   \hline
% (Intercept) & -0.2287 & 0.1594 & -1.43 & 0.1605 \\
%   temp & 0.0076 & 0.0022 & 3.46 & {\color{red}0.0015} \\
%   log(mfgfirms) & -0.0284 & 0.0187 & -1.52 & 0.1389 \\
%   log(popn) & 0.0097 & 0.0226 & 0.43 & 0.6713 \\
%   wind & 0.0216 & 0.0064 & 3.39 & {\color{red}0.0018} \\
%   precip & -0.0018 & 0.0013 & -1.39 & 0.1746 \\
%   raindays & -0.0001 & 0.0006 & -0.21 & 0.8327 \\
%    \hline
% \end{tabular}
% \end{table}
% \end{frame}

% \begin{frame}{Pollution Example}
% \begin{itemize}
% \item Temperature is a significant predictor of $SO_2$ according to the $p$-value.
% \item Lower bound on Bayes Factor
% \[
% 	BF[H_0: H_a]=-e p \log(p)=0.027
% \]
% Here $p$ is the $p$-value.
% \item Strong evidence against that the ccoefficient of Temperature is zero.
% \end{itemize}
% \end{frame}

% \begin{frame}{USair Data}
% \begin{center}
% \includegraphics[scale=0.25]{p_a}
% \end{center}
% \end{frame}

% \begin{frame}{Model Space}
% \begin{center}
% \includegraphics[scale=0.4]{poll_image}
% \end{center}
% \end{frame}

% \begin{frame}{Coefficient}
% \begin{center}
% \includegraphics[scale=0.5]{poll_beta}
% \end{center}
% \end{frame}





\end{document}
